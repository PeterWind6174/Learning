\documentclass[12pt, a4paper]{article}

% --- 1. 核心宏包配置 ---
\usepackage[fontset=mac, heading=true]{ctex}
\usepackage[top=2.5cm, bottom=2.5cm, left=2.5cm, right=2.5cm]{geometry}
\usepackage{graphicx} % 图片支持
\usepackage{amsmath}  % 数学公式
\usepackage{booktabs} % 三线表
\usepackage{caption}  % 图表标题
\usepackage{setspace} % 行间距
\usepackage{enumitem} % 列表
\usepackage{array}    % 表格

% --- 2. 字体设置 (macOS 优化) ---
\setmainfont{Times New Roman}
% 开启伪粗体,确保宋体能显示加粗效果
\setCJKmainfont[AutoFakeBold=true]{Songti SC} 
\setCJKsansfont[AutoFakeBold=true]{Heiti SC} 

% --- 3. 全局段落设置 ---
\setlength{\parindent}{2em} % 首行缩进 2 字符
\setstretch{1.25}           % 行间距 1.25 倍
\pagestyle{plain}           % [关键修改] 强制只有底部页码,去除左上角页眉

% --- 4. 标题格式设置 (宋体 + 编号格式) ---
\ctexset{
  % 一级标题 (例如:一、实验目的)
  section = {
    number = \chinese{section},
    format = \Large\bfseries\rmfamily, 
    aftername = 、,
    indent = 0em,
    beforeskip = 1.5em,
    afterskip = 1em
  },
  % 二级标题 (例如:1、硬件设备)
  subsection = {
    number = \arabic{subsection},
    format = \large\bfseries\rmfamily, 
    aftername = 、,
    indent = 2em,
    beforeskip = 1em,
    afterskip = 0.5em
  }
}

% --- 5. 列表缩进修正 ---
\setlist[enumerate]{leftmargin=2em, labelsep=0.5em, itemindent=2em}
\setlist[itemize]{leftmargin=2em, labelsep=0.5em, itemindent=2em}

% --- 6. 图表标题设置 ---
\captionsetup{labelsep=space}
\captionsetup[table]{position=top}
\captionsetup[figure]{position=bottom}

% --- 7. 封面下划线命令 (关键修改:支持隐藏冒号) ---
% 参数1(可选): 标点符号,默认为冒号
% 参数2: 标签文字
% 参数3: 下划线内容
\newcommand{\coverinfo}[3][ :]{%
    \makebox[4.5em][s]{\textbf{\large #2}}#1 % 标签 + 标点(可选)
    \underline{\makebox[16em][l]{\hspace{0.5em}\large #3}} % 下划线
    \par\vspace{1.2em}
}

\begin{document}

% ================= 封面页 =================
\begin{titlepage}
    \centering
    \vspace*{1cm}
    
    % --- Logo 图片处理 ---
    \IfFileExists{logo1.jpg}{
        \includegraphics[height=3cm]{logo1.jpg}
    }{
        \framebox[3cm]{\parbox[c][3cm][c]{3cm}{\centering \small 校徽/Logo 1}}
    }
    
    \par\vspace{0.5em}
    
    \IfFileExists{logo2.jpg}{
        \includegraphics[height=3cm]{logo2.jpg}
    }{
        \framebox[3cm]{\parbox[c][3cm][c]{3cm}{\centering \small 院徽/Logo 2}}
    }
    
    \vspace{1cm}
    
    {\fontsize{42pt}{50pt}\selectfont \bfseries\rmfamily 电工电子实验报告 \par}
    
    \vspace{3cm}
    
    \begin{center}
        \coverinfo{课程名称}{电子电工实验(一)}
        \coverinfo{实验项目}{实验内容}
        \coverinfo[\phantom{:}]{ }{实验内容} 
        
        \coverinfo{学  院}{通信与信息工程学院}
        \coverinfo{班  级}{B240130}
        \coverinfo{学  号}{B24013026}
        \coverinfo{姓  名}{屠宇航}
        \coverinfo{指导教师}{张潇萧}
        \coverinfo{学  期}{2025-2026学年\quad 第 1 学期}
    \end{center}

\end{titlepage}

% ================= 正文开始 =================
\setcounter{page}{1}

\section{实验目的}

\begin{enumerate}
    \item 目的1
    \item 目的2
    \item 目的3
    \item 目的4
\end{enumerate}

\section{主要仪器设备及软件}

\subsection{硬件设备}
\begin{itemize}
    \item 函数信号发生器
    \item 直流稳压电源
    \item 示波器、万用表
    \item 实验箱、阻容元件及导线若干
\end{itemize}

\subsection{软件环境}
\begin{itemize}
    \item Multisim 14
    \item Matlab R2024b
    \item \LaTeX{} 
\end{itemize}

\section{实验原理/设计过程}

内容

\begin{equation*}
    I = \frac{U_{\text{oc}}}{R_{\text{eq}} + R_L}
\end{equation*}

行内公式$U_{\text{oc}}$ 
\newpage

\section{实验电路图}

\begin{figure}[htbp]
    \centering
    % 智能回退逻辑:有图片显示图片,无图片显示提示框
    \IfFileExists{circuit.png}{
        \includegraphics[width=0.8\textwidth]{circuit.png}
    }{
        \setlength{\fboxsep}{10pt}
        \setlength{\fboxrule}{1pt}
        \fbox{\parbox[c][6cm][c]{0.8\textwidth}{
            \centering 
            \textbf{图片缺失提示} \\ 
            \vspace{1em}
            请将您的电路截图命名为 \texttt{circuit.png} \\
            并放入与 \texttt{.tex} 文件相同的文件夹中。
        }}
    }
    \caption{实验电路图}
\end{figure}

\newpage

\section{实验结果和实验数据分析}

\subsection{任务1}

\begin{table}[htbp]
    \centering
    \caption{戴维南等效参数测量数据}
    \begin{tabular}{cccc}
        \toprule
        \textbf{测量项目} & \textbf{计算值} & \textbf{测量值} & \textbf{误差 ($\Omega$)} \\
        \midrule
        $U_{\text{oc}} (V)$ & 12.0 & 11.95 & 0.42 \\
        $R_{\text{eq}} (k\Omega)$ & 1.0 & 0.99 & 1.0 \\
        \bottomrule
    \end{tabular}
\end{table}

\subsection{任务2}

任务二

\newpage

\section{实验小结}
\subsection{思考题1}
\subsection{思考题2}

\newpage

\section{附录}
仿真截图

\end{document}